\documentclass{article}
\usepackage[margin=1in]{geometry}

\begin{document}

\title{Sensitivity in predicted relative binding free energies from incremental ligand changes within a model binding site}
\date{\today}
\author{Nathan Lim}
\maketitle

\section*{Contributors}
\begin{itemize}
   \item David L. Mobley
   \item Robert Abel
   \item Lingle Wang
   \item Dima Lupyan
\end{itemize}

\section*{Potential Journals}

\section*{Objectives}
\begin{itemize}
   \item Model and capture protein conformational change in the T4 Lysozyme L99A apolar cavity.
   \item Accurately predict relative binding free energies while sampling protein conformational changes.
   \item Compare performance of FEP plus \cite{FEPplus} and LigandFEP.
   \item Compare performance of OPLS2005 and OPLS3 forcefield parameters.
   \item Highlight improvement to FEP/REST \cite{REST2} method by inclusion of protein atoms.
\end{itemize}


\section*{Abstract}
Despite innovations in sampling techniques for molecular dynamics (MD), reliable prediction of protein-ligand binding free energies from MD remains a challenging problem, even in well studied model binding sites like the apolar cavity of T4 Lysozyme L99A\cite{Boyce2009}. 
In this study, we model recent experimental results that show the progressive opening of the binding pocket in response to a series of homologous ligands \cite{Merski2015}. 
Even while using enhanced sampling techniques, we demonstrate that the predicted relative binding free energies (RBFE) are still highly sensitive to the initial protein conformational state. 
Particularly, we highlight the importance of sufficient sampling of protein conformational changes and possible techniques for addressing the issue.

\pagebreak
\section*{Outline}
\section{General Background/Introduction}
   \begin{itemize}
   \item Medicinal chemistry programs typically focus on changes in ligand binding affinity from incremental changes to the ligand.
   \item Focus on how the protein adapts to the changes in the ligand is generally neglected.
   \item T4 L99A is well studied experimentally and computationally. 
      It is frequently used as a model binding site in free energy prediction studies.
   \item In this study, 8 congeneric ligands were investigated, where addition of a single methyl group was used to lengthen the ligand.
   \item Through determination of protein-ligand bound x-ray crystal structures it was revealed T4 lysozyme adopts 3 discrete conformations in response the series of growing ligands.
   \item Consideration of the protein adaptations into discrete conformations may be an important aspect in inhibitor design.
   \end{itemize}

\section{Discrete Conformations and the Ligands}
   \begin{itemize}
   \item T4 L99A contains an engineered apolar cavity which is our binding site of interest.
   \item The 8 congeneric ligands are all apolar and begins with a simple benzene ring. 
      Subsequent ligands are simply addition of a methyl group to generate a growing tail up until n-hexylbenzene.
   \item In response to the growing ligand, the crystal structures show the protein will adopt into 3 conformations aptly named: closed, intermediate, and open.
   \item Primarily, the motion of the protein occurs in the F-helix (residues 107-115), which serves as a sort of gating mechanism into the apolar cavity.
   \item As the ligand tail expands, the F-helix transitions from closed to open, exposing the cavity to the bulk solvent.
   \item From observed electron densities of the F-helix in \cite{Merski2015}-Fig2, the ligands occupy each of the conformations as follows:
   \end{itemize}
   
   \begin{tabular}{|c|c|c|c|}
   \hline
    Ligand           & Closed & Intermediate & Open  \\
    Benzene          & 90\%   & -            &       \\
    Toluene          & 80\%   & 20\%         &       \\
    Ethylbenzene     & 50\%   & 50\%         &       \\
    n-propylbenzene  & 60\%   & 40\%         &       \\
    sec-butylbenzene & 40\%   & 60\%         &       \\
    n-butylbenzene   & 10\%   & 60\%         & 30\%  \\
    n-pentylbenzene  & 30\%   &              & 70\%  \\
    n-hexylbenzene   & 30\%   &              & 70\%  \\
    \hline
    \end{tabular}

    \begin{itemize}
    \item From a protein conformation clustering analysis, shown \cite{Merski2015}, Val111 occupies 3 distinct states iin accordance to the 3 protein conformations.
    \item Visualization of side-chain Val111 from the x-stal structures, shows the backbone alpha-carbon moving ~1.25\AA when transitioning from closed to intermediate, ~3.25\AA intermediate to open, and ~3.50\AA closed to open.
    \item From our MD simulations, it is primarily the repulsive interactions between the ligands and Val111 that drive the F-helix to the open state.
    \end{itemize}

\section{Ligand Binding Affinities}
   \begin{tabular}{|c|c|c|c|}
   \hline
    Ligand           & dGexp (kcal/mol) \\
    benzene          & -5.19            \\
    toluene          & -5.52            \\
    ethylbenzene     & -5.76            \\
    n-propylbenzene  & -6.55            \\
    sec-butylbenzene & N/A              \\
    n-butylbenzene   & -6.70            \\
    n-pentylbenzene  & N/A              \\
    n-hexylbenzene   & N/A              \\
    \hline
    \end{tabular}

    \begin{itemize}
    \item Details of the the ligand binding affinities can be found in the paper \cite{Merski2015}
       \\ Section "Energy of Ligand Binding and Conformational Strain"
    \item Ligands n-pentylbenzene and n-hexylbenzene affinities are inaccessible due to solubility limits.
    \item Generally, as the ligands grow from benzene to n-butylbenzene, the affinity rises linearly.
    \end{itemize}

\section{Methods}
\subsection{FEP Protocols}
   \begin{itemize}
   \item Two FEP protocols developed from Schrodinger were used in this project: FEP+ and LigandFEP.
   \item FEP+ is a fully automated workflow that will plan the perturbation pathways based of the LOMAP mapping algorithm which uses the maximum common substructure (MCS) between any pair of compounds.
   \item LigandFEP, aimed for academics, is limited in the sense that the user must plan each perturbation instead. 
   \item LigandFEP does not use the MCS but rather plans the transformation by minimizing the core RMSD.
   \item Both protocols still use the default relaxation protocol and FEP/REST methodology and were run on GPUs with Desmond.
   \item FEP/REST simulations were performed for 5-55ns depending on the ligand transformation involved. 
   \item In both protocols, forcefield parameters OPLS3 and OPLS2005 were used.
   \item ***INSERT FEP MAPPER IMAGE HERE
   \item Trials including protein atoms in the REST region will be referred to with 'pREST'
   \item Considering the F-helix ranged from residues 107-115, residues selected to include into the REST region were Glu108, Val111, and Gly113 as sort of a start, mid and end point. 
   \end{itemize}

\subsection{Protein/Ligand Preparation}
   \begin{itemize}
   \item Protein structures were taken from PDBS: 4W52,4W53,4W54,4W55,4W56,4W57,4W58,4W59 corresponding to bound structures of benzene, toluene, ethylbenzene, n-propylbenzene, sec-butylbenzene, n-butylbenzene, n-pentylbenzene, and n-hexylbenzene, respectively.
   \item Each simulation will start from either the protein closed state (PDB:4W52) or the open state (PDF: 4W59).
   \item When using the FEP+ protocol, ligand crystal structure positions were used as the starting position of the simulation.
   \item When using the LigandFEP protocol, two options would occur:
      \\If the simulation starts from the protein closed state, the benzene crystal position was used as a reference. 
      The corresponding ligand in the transformation was built by duplicating benzene in place and adding a methyl from the Build/Fragments Toolbar. 
      For ligands with a longer tail, the fragments would be added in the same direction as the crystal structure, but not overlaid or docked via Glide.
      \\If the simulation starts from the protein open state, the n-hexylbenzene crystal position was used as reference. 
      The corresponding ligand in the transformation was built by duplicating n-hexylbenzene in place and deleting a methyl group.
   \item All proteins were prepared and aligned in Maestro using the 'Protein Preparation Wizard' tool and the following settings enabled:
   \begin{itemize}
      \item Preprocess: Assign bond orders, Add hydrogens, Create zero-order bonds to metals, Create disulfide bonds, Cap termini, Delete waters beyond 5\AA from het groups
      \item Refine: Sample water orientations, Use PROPKA pH: 7.0, Remove waters with less than 3 H-bonds to non-waters, and restrained minimization.
   \end{itemize}
   \end{itemize}
   
\section{Discussion}
\subsection{FEP+ vs. LigandFEP}
   \begin{itemize}
   \item Comparisons between the two protocols will be used to show that, although limited, LigandFEP does not result in a difference in performance of accurate predicted relative binding free energies.
   \item Comparisons between FFs will show that the new and improved OPLS3 parameters are give better RBFE predictions for larger transformations.
   \item ***INSERT TABLE COMPARING OPLS2005/OPLS3 BETWEEN TWO PROTOCOLS***
      \\ Both are within the same MUE/RMSE range
   \item Highlight inconsistency in predicted RBFE when the protein starting conformation is varied.
   \item Predicted ddGs are in the wrong direction frequently when starting from protein closed. 
      If we assume smaller to larger ligands should yield favorable (-) ddGs.
   \end{itemize}

\subsection{Small Ligands}
   \begin{itemize}
   \item Small transformations and ligands generally occupy protein closed state.
   \item Equal performance with either FF.
   \item Highlight good case: Toluene to Ethylbenzene with OPLS3
      \begin{itemize}
      \item Small perturbation: Adding one carbon
      \item Both ligands occupy the closed state with some intermediate
      \item Protein starting conformation does not result in a large discrepancy in predicted ddGs.
      \item ***INSERT RMSD GRAPH***
      \item RMSD graph over the lambda=0 and lambda=1 corresponding to the toluene and ethylbenzene end states shows at each time point which conformation (reference to crystal structure) the simulation has the lowest RMSD to.
         \\ Purple = Closed, Teal = Intermediate, and Green = Open.
      \item Highlight good sampling/number of transitions between either state when starting from closed.
      \item Highlight points ~0-1ns are still stuck in open conformation. 240ps relaxation protocol wasn't sufficient to discard few open points.
         \\ Show that this does not have a large impact in the final ddG from sliding time.
      \end{itemize}
   
   \item Highlight not-so-good case: Toluene to n-propylbenzene
      \begin{itemize}
      \item Perturbation involves adding two carbons
      \item Discrepancy from protein starting conformation increases.
      \item Highlight more open points are present (0-2ns) in open runs.
      \end{itemize}
   \item Growing ligands require more time for helix to relax out of open state.
   \item Inclusion of protein residues in REST region should allow for faster transition between states which will give us a lower discrepancy between open/closed runs.
   \item Apply pREST to previous case and show faster relaxation out of open conformation resulting in a lower discrepancy.
   \item ***INSERT COMPARISON BETWEEN NORMAL AND PREST RUNS WITH EXPERIMENTAL ddG ***
   \item Show pREST increases agreement with experiment and lowers error from protein starting conformation.
   \end{itemize}

\subsection{Intermediate Ligands}


   

\pagebreak
\bibliography{references}
\bibliographystyle{unsrt}

\end{document}
