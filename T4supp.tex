\documentclass[T4paper.tex]{subfiles}
\begin{suppinfo}

%Input files for calculations used in paper [DONE]
%Limit output files [DONE]
%Scripts used in analysis [DONE]
%Description for SuppData
%Section on OPLS2005/OPLS3
%Section on LigandFEP/FEP

\section{FEP+ vs. LigandFEP}
   \begin{itemize}
   \item Comparisons between the two protocols will be used to show that, although limited, LigandFEP does not result in a difference in performance of accurate predicted relative binding free energies.
   \item Comparisons between FFs will show that the new and improved OPLS3 parameters are give better RBFE predictions for larger transformations.
   \item ***INSERT TABLE COMPARING OPLS2005/OPLS3 BETWEEN TWO PROTOCOLS***
      \\ Both are within the same MUE/RMSE range
   \item Highlight inconsistency in predicted RBFE when the protein starting conformation is varied.
   \item Predicted ddGs are in the wrong direction frequently when starting from protein closed.
      If we assume smaller to larger ligands should yield favorable (-) ddGs.
   \end{itemize}

\section{OPLS2005 vs. OPLS3}

\section{Case studies}
\subsection{Small Ligands}
   \begin{itemize}
   \item Small transformations and ligands generally occupy protein closed state.
   \item Equal performance with either FF.
   \item Highlight good case: Toluene to Ethylbenzene with OPLS3
      \begin{itemize}
      \item Small perturbation: Adding one carbon
      \item Both ligands occupy the closed state with some intermediate
      \item Protein starting conformation does not result in a large discrepancy in predicted ddGs.
      \item ***INSERT RMSD GRAPH***
      \item RMSD graph over the lambda=0 and lambda=1 corresponding to the toluene and ethylbenzene end states shows at each time point which conformation (reference to crystal structure) the simulation has the lowest RMSD to.
         \\ Purple = Closed, Teal = Intermediate, and Green = Open.
      \item Highlight good sampling/number of transitions between either state when starting from closed.
      \item Highlight points ~0-1ns are still stuck in open conformation. 240ps relaxation protocol wasn't sufficient to discard few open points.
         \\ Show that this does not have a large impact in the final ddG from sliding time.
      \end{itemize}

   \item Highlight not-so-good case: Toluene to n-propylbenzene
      \begin{itemize}
      \item Perturbation involves adding two carbons
      \item Discrepancy from protein starting conformation increases.
      \item Highlight more open points are present (0-2ns) in open runs.
      \end{itemize}
   \item Growing ligands require more time for helix to relax out of open state.
   \item Inclusion of protein residues in REST region should allow for faster transition between states which will give us a lower discrepancy between open/closed runs.
   \item Apply pREST to previous case and show faster relaxation out of open conformation resulting in a lower discrepancy.
   \item ***INSERT COMPARISON BETWEEN NORMAL AND PREST RUNS WITH EXPERIMENTAL ddG ***
   \item Show pREST increases agreement with experiment and lowers error from protein starting conformation.
   \end{itemize}